\documentclass[a4paper]{article}
\usepackage[english]{babel}
\usepackage{enumerate}
\usepackage{amsmath}
\usepackage{graphicx}

\begin{document}

\title{Classifying Seagull Behaviour}
\author{Inge Becht \\ Maarten de Jonge \\ Susanne Marselis \\ Richard Pronk}
\date{\today}
\maketitle

\section{Introduction}
\section{Experimentation}
At the present, it appears that the GPS' measure resolution is inadequate for
any behaviour classification. On a good day there's about 60 measurements, which
simply isn't enough for detecting specific behaviours. Thus we'll have to start
using the accelerometer data early on.
\begin{figure}[htb] 
    \begin{center}
        \makebox[\textwidth] {
            \includegraphics[width=1.2\textwidth]{fig/scatter.png} }
        \caption{Plot of location against time}
        \label{fig:scatter}
    \end{center}
\end{figure}




\section{
Planning
}
\textbf{11-01-2012}
Planning voor eind van de week:
\begin{enumerate}
    \item{Morgen willen we scheiding van land en zee afhebben. Data van een
    bepaalde vogel kunnen scheiden tussen enkel land en enkel zee data.}
    \item{beginnen met segementeren van de accelerometer. Punten om over na te denken:}
    \begin{enumerate}
    \item{op welke schaal kijken we naar de accelerometer data: seconden,
            uren, dagen?}
    \item{hoe gaan we de verschillende segmenten leren? 
            Supervised en zelf dingen met de hand toewijzen? Kmeans
                toepassen en computer clusters laten vinden?  
                uiteindelijk met nearest neighbour nieuwe waardes toewijzen als er
                eenmaal een goed clustermodel gevonden is?}
    \item{
            hebben we alle data nodig(x,y,z) of is een van die al karakteristiek genoeg?
        }
    \end{enumerate}
\end{enumerate}
Vrijdag besloten hoe we dit gaan aanpakken!! 
    


\end{document}
