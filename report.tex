\documentclass[a4paper]{article}
\usepackage[english]{babel}
\usepackage{enumerate}
\usepackage{amsmath}
\usepackage{graphicx}

\begin{document}

\title{Inverse kinematics for the Nao}
\author{Inge Becht \\ Maarten de Jonge \\ Richard Pronk}
\date{\today}
\maketitle

\section{Introduction}
Our research is concerned with making a dynamical kick for the Nao robot. The
Nao is a humanoid robot 58 of cm tall used in the Standard Platform League of
Autonomous Robot Soccer. The kick is a very important part of the soccer game as
having the most goals in a match wins the game. The idea behind a dynamic kick
is that instead of specifically telling for every joint what to do, the Nao
itself will be able to decide given a position of the ball and his own position
how to move his leg to hit the ball without falling over. It is our goal to show
in our last week how the Nao is able to do this, while making a trade-off
between accuracy and stability. If it seems to work properly it will be
integrated in the code of the Dutch Nao Team, the SPL team that represents the
University of Amsterdam, which at this points still only works with setting
keypoints.

\section{Why this project} 
This project covers the one of the most important aspects of the robot soccer
game, the kicking of the ball. Since there is more than one player on the field
there is lots of Interference from the other Naos. Therefore many of the kicks
fail due to the instability of the Nao. Another interesting part of this project
is that this project will Help the Dutch Nao Team in their next competition. The
current keyframe motion is very brittle, with our dynamic kick we will create a
more robust system against these environmental factors.

Overheating is also a problem that Nao’s encounter, here the joints will perform
not as expected or even fail. Using this dynamic system the temperature of the
joints can be used to prevent overheating. This can be done by kicking the ball
not as hard as usual or even kicking with the other leg. Making this dynamic
system for the Nao is a quite difficult task. However there are many papers
concerning this subject, which will be our leading guides to solve this problem. 


\section{Problems, and how to handle them}
The task of kicking a ball requires a couple of things: We need to find a path
for the foot to travel, then we need a way to calculate the joint angles
corresponding to the required position of the foot (inverse kinematics), and all
the while we have to make sure the robot doesn’t fall over.

\begin{enumerate}
  \item The balance problem
  \item The kinematics problem
  \item The kick itself
\end{enumerate}

\subsection{The balance problem}
The balance problem requires knowledge of 2 concepts; the \emph{center of mass},
and the \emph{support polygon}. The center of mass is the weighted average
location of the Nao’s mass, while the support polygon is the location on the
floor over which the center of mass must be located to achieve stability. In the
case of a robot standing on the ground, the support polygon is the convex hull
of the feet touching the ground. In our case, the robot will only be standing on
one foot which should make it rather simple to calculate the support polygon. 

The center of mass is a bit more complex to calculate. It’s defined as the sum
of each component’s \emph{centroid} (its own center of mass) multiplied by its
mass, divided by the total weight (equation \ref{eq:CoM}). This of course
requires each centroid to be described in the same coordinate system.
\begin{align}
  \frac{\sum_i \vec{c}_i m_i} {m}        \label{eq:CoM}
\end{align}
In the Nao’s documentation, each component’s centroid is described relative to
its own coordinate system, and offsets are included to convert between adjacent
components’ coordinate systems. To handle this, we will construct a chain of
transformations to walk through each component while calculating the center of
mass of the entire robot.

\subsection{The kinematics problem}
We want to be able to specify a location for the foot, then have the foot
automatically move there, which means that the we’ll need to be able calculate
the required joint angles. This is known as \emph{inverse kinematics}. There are
two potential solutions to this problem:
\begin{enumerate}
  \item analytically solve the inverse kinematics chain using goniometry (as done by \cite{Graf2009})
\item use a hillclimbing algorithm to approach the desired location over a number of iterations
\end{enumerate}

We’re currently leaning towards the second option, mainly because it’s easier,
more standard, and can deal with unreachable positions (it simply approaches the
closest reachable position).

\subsection{The kick}
With the solutions of the previous two problems, we can start to calculate the
optimal kick trajectory. For this we plan to use a similar system to Naoteam
Humboldt \cite{Muller2011}; we define two measures, one for accuracy and one for
strength, then search for a path which maximizes a certain balance of these two
metrics. 



\section{Required Materials and support} 
To find a solution to our problem statement we need a Nao to experiment on and a
possible simulation environment for at home. All of these materials are
available in the robolab.

Furthermore, there are different papers concerned with the problems we are
trying to tackle, but none of them cover the whole problem statement, and not
all are specifically about the Nao. For example we found a paper about
calculating the center of mass for robots with multiple joints but not tailored
to the problem of the Nao robot.The difficulty for us is to put all this
information together and to make it work specifically on the Nao, and to
document this process at the same time.

\bibliographystyle{plain}
\bibliography{library}
\end{document}
