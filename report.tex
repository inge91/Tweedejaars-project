\documentclass[a4paper]{article}
\usepackage[english]{babel}
\usepackage{enumerate}
\usepackage{amsmath}
\usepackage{graphicx}

\begin{document}

\title{Inverse kinematics for the Nao}
\author{Inge Becht \\ Maarten de Jonge \\ Richard Pronk}
\date{\today}
\maketitle

We used 'a Robust Closed-loop gait for the Standard Platform League Humanoid' to
solve the inverse kinematics problem on the SPL Nao's to try and make a robuster
wall. Here we will talk about the equations that are used in the article and our
interpretation of it

\section*{Equation 1}
Foot2Torso doesn't matter for us because we can use getPosition to find the foot
coordinate in Torso Space.
\begin{align*}    
Foot2Hip = 
\begin{bmatrix}
    1 & 0 & 0 & 0 \\
    0 & 1 & 0 & \frac{ldist}{2}\\
    0 & 0 & 1 & 0 \\
    0 & 0 & 0 & 1 \\
\end{bmatrix}\\
\end{align*}

$ldist$ is the distance between the legs and can be calculated using 
\[getPosition("RAnkleRoll", 0 , True)\]. Substracting the absolute values of y
gives the distance.

\section*{Equation 2}
Now we want to rotate 45 degrees over the x-axis to get the orthogonal
Foot2HipOrth:
\begin{align*}
    Rot_x(\frac{\pi}{4})=
    \begin{bmatrix}
    1&0&0&0\\
    0&cos(\frac{\pi}{4})&-sin(\frac{\pi}{4})&0\\
    0&sin(\frac{\pi}{4})&cos(\frac{\pi}{4})&0\\
    0&0&0&1\\
    \end{bmatrix}
\end{align*}

\section*{Equation 3}
This transformation is inverted so the sum of the transformation matrix
will be the length of the line between the hip and the ankle joint

\section*{Equation 4 and 5}
Now that we know the length between the ankle and the hip, we can calculate all
the angles using the law of cosines. We use this law to calculate the knee angle
on the inside as follows:

\begin{align*}
    cosrules: c^2 = a^2 + b^2 - 2 ab cos\gamma
    a * b * cos(\gamma) =a^2 + b^2 -c^2
    \gamma = arccos(\frac{a^2 + b^2 - c^2}{2 * a * b})
\end{align*}
where a = upperleg, b = lowerleg and $\gamma$ is the joint angle of the knee  

\section*{Equation 6}

\begin{align*}
    \delta_{knee} = \pi - \gamma
\end{align*}
turn the degree of rotation to the outside of the knee.

\section*{Equation 7}
the angle between the ankle and the invisible line from lowerleg to upperleg.
Just the same way calculated as equation 5.
\begin{align*}
    \delta_{footPitch1} = arccos(\frac{a^2 + b^2 - c^2}{2*a*c})
\end{align*}

\section*{Equation 8 and 9}
\begin{align*}
\end{align*}



\end{document}
