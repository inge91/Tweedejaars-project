\documentclass{beamer}
\usepackage[english]{babel}
\usepackage[latin1]{inputenc}
\usepackage{enumerate}
\usepackage{amsmath}
\usepackage{graphicx}
\usetheme{Warsaw}

\AtBeginSection[]
{
  \begin{frame}
    \frametitle{Table of Contents}
    \tableofcontents[currentsection]
  \end{frame}
}

\begin{document}
\title{Designing a dynamic kick for a Nao robot}
\author{Inge Becht \\ Maarten de Jonge \\ Richard Pronk}
\institute{University of Amsterdam}
\date{\today}


\begin{frame}
  \titlepage
\end{frame}

\section{Goal}
\begin{frame}{Goal}
  \begin{itemize}
    \item Creating a dynamic kick instead of a keyfrome motion
    \item Handle external forces without falling
  \end{itemize}
  \pause
  Benefits:
  \begin{itemize}
    \item more stable
    \item omnidirectional
    \item potentially harder
    \item supporting the Dutch Nao Team
  \end{itemize}
\end{frame}

\section{Balancing}
\begin{frame}{Balancing}
  Two components:
  \begin{itemize}
    \item staying balanced during kick trajectory
    \item react to external forces
  \end{itemize}
\end{frame}

\begin{frame}{Proportional-integral-derivative controller}
  Both components use a variant of the PID controller:
  \begin{align*}
    out(t) = K_p e(t) + K_i \int_{0}^{t} e(\tau) d\tau + K_d \frac{d}{dt} e(t)
  \end{align*}
\end{frame}

\begin{frame}{Stripping it down}
  PD- and P-controller:
  \begin{align*}
    out(t) &= K_p e(t) + K_d \frac{d}{dt} e(t) \\
    out(t) &= K_p e(t)
  \end{align*}
\end{frame}

\subsection{Center of Mass Controller}


% Richard, geef het een leuke titel
\subsection{Foot-sensor thing}

\section{Inverse Kinematics}

\section{Trajectory Planning}


\section{Conclusion and future works}

\end{document}
