\documentclass{beamer}
\usepackage[english]{babel}
\usepackage[latin1]{inputenc}
\usepackage{enumerate}
\usepackage{amsmath}
\usepackage{graphicx}
\usetheme{Warsaw}

\AtBeginSection[]
{
  \begin{frame}
    \frametitle{Table of Contents}
    \tableofcontents[currentsection]
  \end{frame}
}

\begin{document}
\title{Designing a dynamic kick for a Nao robot}
\author{Inge Becht \\ Maarten de Jonge \\ Richard Pronk}
\institute{University of Amsterdam}
\date{\today}


\begin{frame}
  \titlepage
\end{frame}

\section{Goal}
\begin{frame}{Goal}
  \begin{itemize}
    \item Creating a dynamic kick instead of a keyfrome motion
    \item Handle external forces without falling
  \end{itemize}
  \pause
  Benefits:
  \begin{itemize}
    \item more stable
    \item omnidirectional
    \item potentially harder
    \item supporting the Dutch Nao Team
  \end{itemize}
\end{frame}

\section{Balancing}
\begin{frame}{Balancing}
  Two components:
  \begin{itemize}
    \item staying balanced during kick trajectory
    \item react to external forces
  \end{itemize}
\end{frame}

\begin{frame}{Proportional-integral-derivative controller}
  Both components use a variant of the PID controller:
  \begin{align*}
    out(t) = K_p e(t) + K_i \int_{0}^{t} e(\tau) d\tau + K_d \frac{d}{dt} e(t)
  \end{align*}
\end{frame}

\begin{frame}{Stripping it down}
  PD- and P-controller:
  \begin{align*}
    out(t) &= K_p e(t) + K_d \frac{d}{dt} e(t) \\
    out(t) &= K_p e(t)
  \end{align*}
\end{frame}

\subsection{Center of Mass Controller}
\begin{frame}{Center of Mass}
  \begin{block}{Balance}
    A humanoid body is balanced when the projection of its center of
mass unto the ground lies within its support polygon
  \end{block}
  Center of mass:
  \begin{align*}
    \frac{\sum_i \vec{c}_i m_i}{M}
  \end{align*}
  Support polygon:
  \begin{itemize}
    \item the convex hull of the contact points with the ground
    \item in case of a robot standing on one leg: the center of its foot
  \end{itemize}
\end{frame}

\begin{frame}{P-Controller}
  \begin{itemize}
    \item CoM calculated through forward kinematics
    \item Uses a simple P-Controller
    \begin{itemize}
      \item seperate for the forwards and sideways directions
    \end{itemize}
    \item Actuation in ankle pitch and roll
  \end{itemize}
\end{frame}

% Richard, geef het een leuke titel
\subsection{Foot-sensor thing}

\section{Inverse Kinematics}
\begin{frame}{Introduction}
  Forward kinematics: Given a set of joint angles, where are the end effectors?
  
  Inverse kinematics: Given desired locations for the end effectors, what is the
  corresponding set of joint angles?
\end{frame}

\begin{frame}{Terminology}
  Let:
  \begin{itemize}
  \item $\theta$ be a $1 \times n$ vector describing the angles of $n$ joints
  \item $\vec{t}$ the vector containing the goal position for each end effector
  \item $\vec{s}$ the vector of the end effectors' current positions
  \item $\vec{e} = \vec{t} - \vec{s}$
  \end{itemize}
\end{frame}

\begin{frame}{Forwards and backwards}
  Viewing end effector locations as a function of the joint angles, we want:
  \begin{align*}
    \vec{t} = \vec{s}(\theta)
  \end{align*}

  \begin{itemize}
    \item not always solvable
    \item can be approached iteratively
  \end{itemize}
\end{frame}

\begin{frame}{Linear approximation}
  \begin{itemize}
  \item Use the first-order derivative of the end effector locations with
    respect to the joint angles as a linear approximation
  \item known as the \emph{Jacobian}
  \end{itemize}
  \begin{block}{The Jacobian}
    \begin{align*}
      J(\theta)_{i, j} &= \frac{\partial s_i}{\partial \theta_j} \\
      \frac{\partial s_i}{\partial \theta_j} &= v_j \times (s_i - p_j)
    \end{align*}
  \end{block}
\end{frame}

\begin{frame}{Iterative solution}
  Linear approximation:
  \begin{align*}
    \Delta \vec{s} = J \Delta\theta
  \end{align*}
  We're interested in:
  \begin{align*}
    \Delta \vec{e} = J \Delta\theta
  \end{align*}
  and thus:
  \begin{align*}
    \Delta \theta = J^{-1} \vec{e}
  \end{align*}
  \pause
  Because it's only a linear approximation, place an upper limit on the length
  of $\vec{e}$.
\end{frame}

\begin{frame}{Inverting the Jacobian}
  \begin{itemize}
    \item Jacobian usually can't be inverted
    \item In our case: $3\times4$ matrix
  \end{itemize}
  \pause
  Three common methods of approximating the inverse:
  \begin{itemize}
    \item Jacobian transpose: $J^{-1} \approx \alpha J^T$
    \item Moore-Penrose pseudoinverse: $J^{-1} approx J^{+}$
    \item Damped Least Squares: $(J^T J  + \lambda^2 I)^1 J^T$
  \end{itemize}
\end{frame}

\section{Trajectory Planning}

\section{Conclusion and future works}

\end{document}
