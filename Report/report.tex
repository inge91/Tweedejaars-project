\documentclass[a4paper]{article}
\usepackage[english]{babel}
\usepackage{enumerate}
\usepackage{amsmath}
\usepackage{graphicx}

\begin{document}

\title{Designing a Dynamic Kick for the Nao Robot}
\author{Inge Becht \\ Maarten de Jonge \\ Richard Pronk}
\date{\today}
\maketitle

\section{Introduction}
Our research is concerned with making a dynamic kick for the Nao robot. The
Nao is a 58 cm tall humanoid robot used in the Standard Platform League of
Autonomous Robot Soccer. The kick is a very important part of the soccer game as
having the most goals in a match wins the game. The idea behind a dynamic kick
is that instead of specifically telling for every joint what to do, the Nao
itself will be able to decide given a position of the ball and his own position
how to move his leg to hit the ball without falling over. It is our goal to show
in our last week how the Nao is able to do this, while making a trade-off
between accuracy and stability. If it seems to work properly it will be
integrated in the code of the Dutch Nao Team, the SPL team that represents the
University of Amsterdam, which at this points still only 
define motions with setting the 
joints manually (by keyframe motions).

\section{Motivation for this project} 
While the Dutch Nao Team has achieved some successes with their simple
keyframe motions, having a dynamic kick should still be desired as it tackles
some different problems that arise when learning robots to play soccer. 


\subsection{Stability}
Since there is more than one player on
the field  chasing after the ball,
there is lots of interference from other Naos while kicking. Due to the
instability of the Nao this results in falling down quite often, even more
so when there is no compensation for it when executing a motion. A dynamic
kick keeps track of the stability and will execute a kick that keeps the nao
most stable.
\\\\
Overheating is also a problem for Naos. When this occurs the joints will perform
not as expected or even fail. Using this dynamic system the temperature of the
joints can be used to prevent overheating. The executed kick will depend less
on force so the robot will spare its joints.

\subsection{Harder kicks and better stability}
Being able to define when a kick is stable enough to execute we can make a
trade-off between how far the ball is kicked and the stableness of the
robot, resulting in harder kicks then when using keyframe motions.

\subsection{Helping the Dutch Nao Team}
 Working on this project will help the Dutch Nao Team in their competition. The
current keyframe motion used by DNT is very brittle, with our dynamic kick we will create a
more robust kick that can give our universities team a bigger chance of
winning in competitions. 
This integration makes this project a valuable activity in the long run, and not
just a one time experiment.

\section{Problems, and how to handle them}
The task of kicking a ball requires a couple of things: We need to find a path
for the foot to travel, then we need a way to calculate the joint angles
corresponding to the required position of the foot (inverse kinematics), and all
the while we have to make sure the robot doesn't fall over.

\subsection{The balance problem}
The balance problem requires knowledge of 2 concepts; the \emph{center of mass},
and the \emph{support polygon}. The center of mass is the weighted average
location of the Nao’s mass, while the support polygon is the location on the
floor over which the center of mass must be located to achieve stability. In the
case of a robot standing on the ground, the support polygon is the convex hull
of the feet touching the ground. In our case, the robot will only be standing on
one foot which should make it rather simple to calculate the support polygon. 

The center of mass is a bit more complex to calculate. It's defined as the sum
of each component’s \emph{centroid} (its own center of mass) multiplied by its
mass, divided by the total weight (equation \ref{eq:CoM}). This of course
requires each centroid to be described in the same coordinate system.
\begin{align}
  \frac{\sum_i \vec{c}_i m_i} {m}        \label{eq:CoM}
\end{align}
In the Nao’s documentation, each component’s centroid is described relative to
its own coordinate system, and offsets are included to convert between adjacent
components’ coordinate systems. To handle this, we will construct a chain of
transformations to walk through each component while calculating the center of
mass of the entire robot.

\subsection{The kinematics problem}
We want to be able to specify a location for the foot, then have the foot
automatically move there, which means that the we'll need to be able calculate
the required joint angles. This is known as \emph{inverse kinematics}. There are
two potential solutions to this problem:
\begin{enumerate}
  \item analytically solve the inverse kinematics chain using goniometry (as done by \cite{Graf2009})
\item use a hillclimbing algorithm to approach the desired location over a number of iterations
\end{enumerate}

We're currently leaning towards the second option, mainly because it’s easier,
more standard, and can deal with unreachable positions (it simply approaches the
closest reachable position).

\subsection{The kick}
With the solutions of the previous two problems, we can start to calculate the
optimal kick trajectory. For this we plan to use a similar system to Nao team
Humboldt \cite{Muller2011}; we define two measures, one for accuracy and one for
strength, then search for a path which maximizes a certain balance of these two
metrics. 



\section{Required materials and support} 
To find a solution to our problem statement we need a Nao to experiment on and a
possible simulation environment for at home. All of these materials are
available in the robolab.

Furthermore, there are different papers concerned with the problems we are
trying to tackle, but none of them cover the whole problem statement, and not
all are specifically about the Nao. For example we found a paper about
calculating the center of mass for robots with multiple joints\cite{Cotton2008} but not tailored
to the connection of joints of a  Nao robot.The difficulty for us is to put all this
information together and to make it work specifically on the Nao, and to
document this process at the same time. Most of the problems have been covered
in previous courses, but it still will require some insight in achieving our set
goals.


\bibliographystyle{plain}
\bibliography{library}
\end{document}
